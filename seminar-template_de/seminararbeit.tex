\documentclass[deutsch]{llncs}

\usepackage{url}
\usepackage{graphicx}
\usepackage{listings}
\usepackage{verbatim}
\usepackage{listings}
\lstset{numberbychapter=false}
\usepackage[lined,linesnumbered,german]{algorithm2e}
\SetAlCapFnt{\small}
\SetAlCapNameFnt{\small}
\usepackage{tikz}
\usetikzlibrary{arrows,automata}
\usepackage{xspace}
%\usepackage{bibtex}

% for german seminar theses
\usepackage[utf8]{inputenc}
\usepackage[ngerman]{babel}
\usepackage{csquotes}
\usepackage[abbreviate=false,maxbibnames=99,backend=bibtex]{biblatex}
\bibliography{referenzen}

\usepackage{hyperref}

\setcounter{secnumdepth}{2}
\setcounter{tocdepth}{3}

% define custom macros for specific formats or names
\newcommand{\uml}[1]{\texttt{#1}\xspace}
\newcommand{\cd}{\textsf{Klassendiagramm}\xspace}

% Numeriere 3 Ebenen tief (bis subsection)
\setcounter{secnumdepth}{2}

% make a proper TOC despite llncs
\setcounter{tocdepth}{2}
\makeatletter
\renewcommand*\l@author[2]{}
\renewcommand*\l@title[2]{}
\makeatletter

\begin{document}
\def\abstractname{Kurzfassung.}

\pagestyle{plain}
\pagenumbering{roman}

\title{Virtuelle und erweiterte Real‚ität\thanks{Diese Arbeit wurde im Rahmen der LVA ``Wissenschaftliches Arbeiten'' (188.925) im WS18 erstellt.}}


%&&&&&&&&&&&&&&&&&&&&&&&&&&&&&&&&&&&&&&&&&&&&&&&&&&&&&&&&&&&&&&&&&&&&&&&&
% Name and address of the author
%&&&&&&&&&&&&&&&&&&&&&&&&&&&&&&&&&&&&&&&&&&&&&&&&&&&&&&&&&&&&&&&&&&&&&&&&
%\author{Max Mustermann}

%\institute{Technische Universität Wien\\ Bachelorstudium Wirtschaftsinformatik\\ \email{max.mustermann@tuwien.ac.at} \\ Matrikelnr.: 0123456}

%&&&&&&&&&&&&&&&&&&&&&&&&&&&&&&&&&&&&&&&&&&&&&&&&&&&&&&&&&&&&&&&&&&&&&&&&
% Example for more than one authors
%&&&&&&&&&&&&&&&&&&&&&&&&&&&&&&&&&&&&&&&&&&&&&&&&&&&&&&&&&&&&&&&&&&&&&&&&
\author{Barbara Elias\inst{1} \and Wang Yi\inst{2}}

\institute{Technische Universität Wien\\ Bachelorstudium Medizinische Informatik\\ \email{e1028094@student.tuwien.ac.at} \\ Matrikelnr.: 1028094
\and
Technische Universität Wien\\ Bachelorstudium Wirtschaftsinformatik\\ \email{e1633407@student.tuwien.ac.at} \\ Matrikelnr.: 01633407}

\maketitle

% reset footnote counter in case of multiple authors
\setcounter{footnote}{0}

\begin{abstract}
Diese Arbeit beschäftigt sich mit dem Thema virtuelle und erweiterte Realität insbesondere zu Ausbildungszwecken. Im Speziellen werden Publikationen von Mag. Dr. Hannes Kaufmann zur näheren Erarbeitung der Nutzung erweiterter Realität in Hinblick auf den Geometrieunterricht herangezogen. Seine Arbeiten zielen darauf ab, Geometrie begreifbarer zu machen, als es mit der Konstruktion mittels Papier und Bleistift möglich ist. Basierend auf schon vorhandener Geometriesoftware wurden VR-Anwendungen konstruiert, die Mathematik verständlicher machen sollen. Die Idee dabei ist, den Schülerinnen und Schülern die Möglichkeit zu geben, mittels VR-Brille um dreidimensionale Objekte zu gehen und diese von neuen, ungeahnten Perspektiven erkennen zu können und damit ein besseres Verständnis für räumliche Geometrie zu bekommen. 
\end{abstract}

%&&&&&&&&&&&&&&&&&&&&&&&&&&&&&&&&&&&&&&&&&&&&&&&&&&&&&&&&&&&&&&&&&&&&&&&&
% Table of contents
% Activate or deactivate this according to the guideline instructor
%&&&&&&&&&&&&&&&&&&&&&&&&&&&&&&&&&&&&&&&&&&&&&&&&&&&&&&&&&&&&&&&&&&&&&&&&
\tableofcontents
\newpage

\pagenumbering{arabic}

\section{Einleitung}
\label{sec:intro}
Die Geschichte der virtuellen Realität reicht länger zurück als man auf den ersten Blick glauben möchte. Schon analoge Systeme der virtuellen Realität lassen sich finden, diese sind zu Trainingszwecken im militärischen Bereich eingesetzt worden \cite{1}. 
Mittlerweile findet man zahlreiche Anwendungen der virtuellen und erweiterten Realität auch im Alltag, beispielsweise als Flugsimulatoren, in der Spieleindustrie (Konsolenspiele), in der Medizin zur Unterstützung bei Operationen oder aber um Lernstoff im wahrsten Sinn des Wortes ``begreifbar'' zu machen. 
Diese Arbeit beschäftigt sich mit dem Thema virtuelle und erweiterte Realität, insbesondere damit, wie sich Anwendungen der erweiterten und virtuellen Realität in der Aus- und Weiterbildung nutzen lassen. 
Zunächst gilt es zu erklären, was sich hinter den Begriffen ``virtuelle Realität'' und ``erweiterte Realität'' verbirgt. 
\noindent \\
Unter virtueller Realität (Virtual Reality, VR) versteht man eine computergenerierte Welt, die in Echtzeit von ihrem Benutzer erforscht und erlebt werden kann und alle physikalischen Eigenschaften wahrheitsgetreu abbilden kann. 
Beispielsweise findet man sogenannte VR-Brillen, mit denen man Computerspiele ganz neu erleben kann.
Augmented Reality (AR, bzw. erweiterte Realität) vermischt virtuelle Realität und physische Realität und wird daher auch als ``mixed reality'' bezeichnet, im Gegensatz zur virtuellen Realität ist man als Nutzer hier nicht von seiner Umwelt abgegrenzt.
Als Beispiele für erweiterte Realität kann hier Googles ``glass'' genannt werden. \\
\noindent \\
Die restliche Arbeit gliedert sich in 4 Kapitel wie folgt: 
Kapitel 2 gibt einen Einblick in State of the Art von VR und AR im. In Kapitel 3 findet sich ein Einblick in die Arbeiten von Hannes Kaufmann, während Kapitel 4 eine Zusammenfassung und Kapitel 5 das Literaturverzeichnis enthält.

\section{Einsatz in der Aus- und Weiterbildung}
\label{sec:typo}
State of the Art - wie wendet man VR und AR heutzutage an.. to be continued \\
\noindent \\
Da die Zeit, in der sich Hannes Kaufmann mit virtueller und erweiterter Realität im Unterricht beschäftigt hat, schon etwas zurückliegt, geht dieses Kapitel auf den aktuellen Entwicklungsstand und State of the Art ein. \\
Schülerinnen und Schüler stellen immer mehr den Sinn dessen, was sie tagtäglich in der Schule lernen sollen, in Frage. Lehrerinnen und Lehrer wiederum sind oft überfordert damit, wie sie den vorgeschriebenen Unterrichtsstoff zeitgemäß nahe bringen können. Um Schüler\_innen und Studierende zeitgemäß zu unterrichten, müssen sich neue Methoden etablieren, bzw. haben sich bereits etabliert, da man sonst Gefahr läuft, dass Studierende die Lust am Lernen verlieren. Heutzutage geht man weg von Frontalvorträgen, die isoliert von ihrem Kontext vorgetragen werden, hin zum Einsatz von neuen Medien wie beispielsweise virtuelle Realität im Unterricht. Mehrere Studien [QUELLEN EINFÜGEN NICHT VERGESSEN] zeigen, dass man damit den Einsatz und die Begeisterungsfähigkeit von Studierenden massiv heben kann.
\cite{2}
[Can Virtual Reality Help Children Learn Mathematics Better? The Application of VR Headset in Children’s Discipline Education].


\section{Virtuelle und erweiterte Realität in der Geometrie}
Mag. Dr. Hannes Kaufmann hat seinen Forschungsschwerpunkt auf AR und VR gelegt und einige wissenschaftliche Arbeiten dazu verfasst. \\
\noindent \\
Erfahrungsgemäß haben einige Schülerinnen und Schüler Probleme damit, sich geometrische Objekte, dargestellt auf Papier oder einer Schultafel vorzustellen und anhand dieser Skizzen mathematische Problemstellungen zu begreifen und zu lösen. Mit dem Einsatz von VR bzw. AR soll dieses Problem gelöst werden

\label{sec:typo}

\subsection{Designing Immersive Virtual Reality for Geometry Education}
\label{subsec:}

3.2  In dem Artikel ,, Designing Immersive Virtual Reality for Geometry Education “ handelt es sich um den Entwurf von Construct3D. Dabei werden die Motivation und iteratives Design Prozess beschrieben und die Verbesserung der Benutzerschnittstelle dargestellt. Schlussendlich werden die Bewertungsergebnisse von Construct3D im Hinblick auf die Zielgruppe, nämlich Lehrenden und Lernenden, präsentiert. [S.2] \\
Das Ziel von Construct3D ist nicht ein professionelles 3D-Modellierungswerkzeug zu erstellen, sondern ein einfaches, intuitives und dynamisches Konstruktionswerkzeug in einer immersiven Umgebung für Bildungszwecke zu kreieren. Eine grundlegende Eigenschaft von solcher dynamischer Geometrie-Software ist, dass das dynamische Verhalten einer Konstruktion durch interaktives Bewegen der einzelnen definierten Elemente wie z.B. Eckpunkte eines starren Körpers ersichtlich ist. Das pädagogische Ziel ist zu überprüfen, ob das Arbeiten direkt im 3D-Raum ein besseres und schnelleres Verständnis von komplexen räumlichen Problemen und deren Zusammenhängen ermöglichen kann als die traditionellen Lehrmethoden. [S.1]\\
Eine der grundlegenden Entscheidungen bei der Entwicklung von Construct3D basiert auf der Erkenntnis, dass die genaue Konstruktion von Koordinaten im 3D-Raum nur schwer zu erreichen ist, wenn sie direkt nach den sechs Freiheitsgraden manipuliert wird. Beispielsweise Verfolgung von Ungenauigkeiten, mangelnde Koordination von Händen und Augen, und Schwierigkeiten bei der genauen Lokalisierung eines 3D-Punktes dargestellt mit einem fest fokussierten stereoskopischen Head-Mounted-Display (HMD) machen schnelle und präzise Eingabe schwierig. Daher wird die Eingabe der Benutzenden auf Dimensionen eingeschränkt. Im Gegensatz zu traditioneller Erziehung sind Koordinaten als Positionsangaben im Raum einer dynamischen Bauumgebung von sehr geringer Bedeutung. Es ist dennoch nach wie vor wichtig, über leistungsfähige Snap-Funktionen zu verfügen, um Objekte in korrekten Beziehungen zueinander zu konstruieren. [S.2]\\
Das für Construct3D verwendete Setup unterstützt zwei kooperierende Benutzenden mit stereoskopischen Durchsicht-HMDs (Sony Glasstron D100BE), die einen gemeinsamen virtuellen Raum bereitstellt. Die Benutzenden interagieren im System mit Stiften und Block-Stützen. Da beide Benutzenden die gleichen virtuellen Objekte sowie den Stift und das Menü der anderen Person sehen können, ist es den Benutzenden möglich gegenseitig zu helfen. Dabei werden Kopf und Hände mit einem ARTTrack1 Tracking-System verfolgt. Eine stationäre Kamera ist ebenfalls vorhanden, um Zuschauenden einen zusätzlichen erweiterten Blickwinkel zu bieten oder für Videodokumentationen. Die Softwareplattform Studierstube wird von Construct3D als Laufzeitumgebung und für die Mehrbenutzersynchronisation benutzt. [S.2 f.]\\
Die aktuelle Version von Construct3D stellt Funktionen für die Konstruktion von 3D-Punkten und geometrischen Objekten (wie beispielsweise Kugeln und nicht-uniforme rationale B-Splines) zur Verfügung. Sie bietet auch planare und räumliche geometrische Operationen an diesen Objekten (wie beispielsweise Boolesche Operationen und Drehungen) sowie Messungen und Strukturierung von Elementen in Ebenen. [S. 3]\\
Das Menüsystem von Construct3D ist auf eine tragbare Stift- und Bedienoberfläche, das Personal Interaction Panel, abgebildet. Es gibt fünf Untermenüs, die über Registerkarten zugänglich sind. Irrelevante Funktionen oder Widgets können deaktiviert werden. Somit können die Lernenden auf die eigentliche Aufgabe konzentrieren. Dadurch kann der kognitive Aufwand für die Nutzung der Anwendung reduziert werden. [S.3]\\
Zu den Visualisierungstechniken, die in Construct3D verwendet werden, gehören die Verwendung von Transparenz, um das Verständnis der Benutzenden für die Konstruktion zu verbessern, die Farbcodierung, um zwischen den Beiträgen mehrerer Benutzenden zu unterscheiden, die Trennung in Ebenen, um die semantische Strukturierung einer Konstruktion zu unterstützen, und die automatische Vorschau neuer Objekte. Obwohl diese Techniken Szenenverarbeitung und grafisches Rendern trotz des einfachen Aussehens der Anwendung ziemlich teuer machen, ist diese zusätzliche Rechenleistung wert, da die Benutzerfreundlichkeit nach der Einführung dieser Funktionen verbessert wird. [S.3]\\
In der ersten Version von Construct3D wurde ein Schieberegler implementiert, der den Benutzenden die Möglichkeit gab, die Transparenz von Objekten selbst zu ändern. Dies war nicht zufriedenstellend, da viele Objekte nach einer Reihe von Transparenzveränderungen unterschiedliche Transparenzen hatten, was zu Verwirrung führte. Um eine konsistente Lernumgebung zu schaffen, wurden feste Transparenzwerte für alle Objekte und Farbschemata entworfen, sodass Objekte hinter mehr als zwei weiteren überlappenden 3D-Objekten noch zu sehen sind. Komplexe Objekte sind jedoch undurchsichtig gezeichnet. Es ist aber möglich, sie einzeln in den Wireframe-Modus zu schalten. [S.4]\\
Um geometrische Inhalte zu strukturieren wurde neben der Kodierung von Benutzerinformationen im Farbschema (blau, orange, grün und rot) auch Wert daraufgelegt, visuelle Informationen über aktive und inaktive Ebenen zu haben, indem die Inaktiven in einem entsättigten Stil dargestellt werden. Dies wird in der Abbildung 1 illustriert. Während aktive Objekte sich im rechten oberen Bereich jedes Bildes befinden, sind inaktiven Objekten im jeweiligen linken unteren Ecken zu sehen. Alle Screenshots in der ersten Zeile zeigen einen Vergleich zwischen desselektierten und inaktiven Objekten. In der zweiten Zeile werden die Ausgewählten mit inaktiven Ebenen verglichen. [S.4 f.]\\
 
Abbildung 1: Drei Construct3D Unterräume mit Farben für ausgewählte / abgewählte / aktive und inaktive Ebenen.[S.5]
Da die Implementierung des Farbschemas für die Farbe pro benutzende Person und die Anzeige von Ebenen sich als problematischer als ursprünglich erwartet erwies, wurde jedem Primitiv ein anderes Material zugewiesen. Insgesamt wurden mehr als 140 verschiedene Materialien für die Objekte entworfen, um ein einzigartiges Aussehen und Gefühl zu erzeugen. 6 Lichter wurden zu der Szene hinzugefügt, um einheitliche Beleuchtungsbedingungen unabhängig von der Position der Benutzenden in der virtuellen Umgebung zu erzeugen.[S.5]
Wenn der Stift der benutzenden Person einem Objekt am nächsten liegt, wird dieses Objekt mit ,,wireframe grid“, was wie ein Netz ausschaut, hervorgehoben. Somit weiß die Person, dass das Objekt ausgewählt werden kann. Beim Nahekommen einem Punkt kann dieser auch gezogen werden. Um dies zu zeigen, leuchtet die Farbe vom Punkt (Siehe Abbildung 2). Wenn eine Konstruktion mit den Eingabeparametern funktioniert, erfolgt sofort eine visuelle Rückmeldung. Die Vorschaufunktion wird dann aktiviert, indem der Stift über ein Widget bewegt wird.[S.5 f.]
 
Abbildung 2 Links: Vorschaue eines Zylinders. Rechts oben: Ein hervorgehobener Punkt.
Rechts unten: Ein Punkt, der gezogen werden kann. [S.6]
Basierend auf den Feedbacks aus vielen Versuchen mit Oberschüler\_innen wird Construct3D über 3 Jahre kontinuierlich verbessert. Im Jahr 2004 wurde eine größere Studie zu Construct3D mittels Interviews und den standardisierten ISONORM 9241/10 Usability-Fragebogen mit 9 Schülern und 6 Schülerinnen durchgeführt. Da dieser Inhalt im nächsten Artikel von Dr. Kaufmann ,, Summary of Usability Evaluations of an Educational Augmented Reality Application“(Unterpunkt 5.3) ausführlicher beschrieben wird, wird es hier nicht näher behandelt. Zusammengefasst, wurde der Arbeitsstil mit Construct3D von Lernenden als spaß-machend empfunden, und Construct3D als geeignet für das Erlernen der Geometrie subjektiv bewertet, obwohl Probleme wie Kopfschmerzen und unvorhersehbare Reaktionszeit der Software dabei aufgetaucht waren.[S.6 f.]









\section{Zusammenfassung}
\label{sec:typo}
In dieser Zusammenfassung .... 

\section{Literaturverzeichnis}
\label{sec:bib}

\subsection{Literatursuche}
\label{subsec:search}


\subsection{BibTeX}
\label{subsec:bibtex}
[1] https://www.vrnerds.de/die-geschichte-der-virtuellen-realitaet/
\cite{2 http://virtualrealityforeducation.com/wp-content/uploads/2018/06/HuAu_Lee_2017_VRinEd.pdf}

\printbibliography

\end{document}
